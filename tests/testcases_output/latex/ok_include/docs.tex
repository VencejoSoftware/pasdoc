\documentclass{report}
\usepackage{hyperref}
% WARNING: THIS SHOULD BE MODIFIED DEPENDING ON THE LETTER/A4 SIZE
\oddsidemargin 0cm
\evensidemargin 0cm
\marginparsep 0cm
\marginparwidth 0cm
\parindent 0cm
\setlength{\textwidth}{\paperwidth}
\addtolength{\textwidth}{-2in}


% Conditional define to determine if pdf output is used
\newif\ifpdf
\ifx\pdfoutput\undefined
\pdffalse
\else
\pdfoutput=1
\pdftrue
\fi

\ifpdf
  \usepackage[pdftex]{graphicx}
\else
  \usepackage[dvips]{graphicx}
\fi

% Write Document information for pdflatex/pdftex
\ifpdf
\pdfinfo{
 /Author     (Pasdoc)
 /Title      ()
}
\fi


\begin{document}
\label{toc}\tableofcontents
\newpage
% special variable used for calculating some widths.
\newlength{\tmplength}
\chapter{Introduction}
\label{ok_include_intro}
\index{ok{\_}include{\_}intro}


Demo that you can also split introduction and conclusion using @include tag:

\label{Sec1}
\section{First section in ok{\_}include{\_}intro.txt file}


First section dummy text.

\label{Sec2}
\section{Second section in ok{\_}include{\_}intro{\_}include.txt file}


Second section dummy text.

\chapter{Unit ok{\_}include}
\label{ok_include}
\index{ok{\_}include}
\section{Description}
This is a test of @include tag.\hfill\vspace*{1ex}



Behold included file ok{\_}include{\_}1.txt: Blah blah blah. This is file \begin{ttfamily}ok{\_}include{\_}1.txt\end{ttfamily}. This is file \begin{ttfamily}ok{\_}include{\_}2.txt\end{ttfamily}. 

Behold file ok{\_}include{\_}1.txt that is included for the 2nd time here: Blah blah blah. This is file \begin{ttfamily}ok{\_}include{\_}1.txt\end{ttfamily}. This is file \begin{ttfamily}ok{\_}include{\_}2.txt\end{ttfamily}. 

Behold file ok{\_}include{\_}1.txt that is included for the 3rd time here, and this time it's inside @bold: \textbf{Blah blah blah. This is file \begin{ttfamily}ok{\_}include{\_}1.txt\end{ttfamily}. This is file \begin{ttfamily}ok{\_}include{\_}2.txt\end{ttfamily}. }

Take a look at \begin{ttfamily}Introduction\end{ttfamily}(\ref{ok_include_intro}) too.
\end{document}
