\documentclass{report}
\usepackage{hyperref}
% WARNING: THIS SHOULD BE MODIFIED DEPENDING ON THE LETTER/A4 SIZE
\oddsidemargin 0cm
\evensidemargin 0cm
\marginparsep 0cm
\marginparwidth 0cm
\parindent 0cm
\textwidth 16.5cm

\ifpdf
  \usepackage[pdftex]{graphicx}
\else
  \usepackage[dvips]{graphicx}
\fi

\begin{document}
% special variable used for calculating some widths.
\newlength{\tmplength}
\chapter{Unit ok{\_}operator{\_}test}
\section{Description}
Operator overloads Delphi and FPC
\section{Overview}
\begin{description}
\item[\texttt{\begin{ttfamily}TDelphiRec\end{ttfamily} Record}]Operator overloads declared within a record (Delphi 2006+)
\item[\texttt{\begin{ttfamily}TMyClass\end{ttfamily} Class}]In this case, "Operator" is used as a normal Delphi identifier
\item[\texttt{\begin{ttfamily}TMyType\end{ttfamily} Record}]
\item[\texttt{\begin{ttfamily}TMyType2\end{ttfamily} Record}]
\end{description}
\begin{description}
\item[\texttt{:=}]
\item[\texttt{+}]
\item[\texttt{-}]
\item[\texttt{*}]
\item[\texttt{/}]
\item[\texttt{**}]
\item[\texttt{=}]
\item[\texttt{{$<$}}]
\item[\texttt{{$>$}}]
\item[\texttt{{$<$}=}]
\item[\texttt{{$>$}=}]
\item[\texttt{or}]
\item[\texttt{and}]
\item[\texttt{xor}]
\end{description}
\section{Classes, Interfaces, Objects and Records}
\subsection*{TDelphiRec Record}
\subsubsection*{\large{\textbf{Description}}\normalsize\hspace{1ex}\hfill}
Operator overloads declared within a record (Delphi 2006+)\subsubsection*{\large{\textbf{Methods}}\normalsize\hspace{1ex}\hfill}
\paragraph*{Add}\hspace*{\fill}

\begin{list}{}{
\settowidth{\tmplength}{\textbf{Description}}
\setlength{\itemindent}{0cm}
\setlength{\listparindent}{0cm}
\setlength{\leftmargin}{\evensidemargin}
\addtolength{\leftmargin}{\tmplength}
\settowidth{\labelsep}{X}
\addtolength{\leftmargin}{\labelsep}
\setlength{\labelwidth}{\tmplength}
}
\begin{flushleft}
\item[\textbf{Declaration}\hfill]
\begin{ttfamily}
public class operator Add(a, b: TDelphiRec): TDelphiRec;\end{ttfamily}


\end{flushleft}
\par
\item[\textbf{Description}]
Addition of two operands of type TDelphiRec

\end{list}
\paragraph*{Subtract}\hspace*{\fill}

\begin{list}{}{
\settowidth{\tmplength}{\textbf{Description}}
\setlength{\itemindent}{0cm}
\setlength{\listparindent}{0cm}
\setlength{\leftmargin}{\evensidemargin}
\addtolength{\leftmargin}{\tmplength}
\settowidth{\labelsep}{X}
\addtolength{\leftmargin}{\labelsep}
\setlength{\labelwidth}{\tmplength}
}
\begin{flushleft}
\item[\textbf{Declaration}\hfill]
\begin{ttfamily}
public class operator Subtract(a, b: TDelphiRec): TDelphiRec;\end{ttfamily}


\end{flushleft}
\par
\item[\textbf{Description}]
Subtraction of type TDelphiRec

\end{list}
\paragraph*{Implicit}\hspace*{\fill}

\begin{list}{}{
\settowidth{\tmplength}{\textbf{Description}}
\setlength{\itemindent}{0cm}
\setlength{\listparindent}{0cm}
\setlength{\leftmargin}{\evensidemargin}
\addtolength{\leftmargin}{\tmplength}
\settowidth{\labelsep}{X}
\addtolength{\leftmargin}{\labelsep}
\setlength{\labelwidth}{\tmplength}
}
\begin{flushleft}
\item[\textbf{Declaration}\hfill]
\begin{ttfamily}
public class operator Implicit(a: Integer): TDelphiRec;\end{ttfamily}


\end{flushleft}
\par
\item[\textbf{Description}]
Implicit conversion of an Integer to type TDelphiRec

\end{list}
\paragraph*{Implicit}\hspace*{\fill}

\begin{list}{}{
\settowidth{\tmplength}{\textbf{Description}}
\setlength{\itemindent}{0cm}
\setlength{\listparindent}{0cm}
\setlength{\leftmargin}{\evensidemargin}
\addtolength{\leftmargin}{\tmplength}
\settowidth{\labelsep}{X}
\addtolength{\leftmargin}{\labelsep}
\setlength{\labelwidth}{\tmplength}
}
\begin{flushleft}
\item[\textbf{Declaration}\hfill]
\begin{ttfamily}
public class operator Implicit(a: TDelphiRec): Integer;\end{ttfamily}


\end{flushleft}
\par
\item[\textbf{Description}]
Implicit conversion of TDelphiRec to Integer

\end{list}
\paragraph*{Explicit}\hspace*{\fill}

\begin{list}{}{
\settowidth{\tmplength}{\textbf{Description}}
\setlength{\itemindent}{0cm}
\setlength{\listparindent}{0cm}
\setlength{\leftmargin}{\evensidemargin}
\addtolength{\leftmargin}{\tmplength}
\settowidth{\labelsep}{X}
\addtolength{\leftmargin}{\labelsep}
\setlength{\labelwidth}{\tmplength}
}
\begin{flushleft}
\item[\textbf{Declaration}\hfill]
\begin{ttfamily}
public class operator Explicit(a: Double): TDelphiRec;\end{ttfamily}


\end{flushleft}
\par
\item[\textbf{Description}]
Explicit conversion of a Double to TDelphiRec

\end{list}
\subsection*{TMyClass Class}
\subsubsection*{\large{\textbf{Hierarchy}}\normalsize\hspace{1ex}\hfill}
TMyClass {$>$} TObject
\subsubsection*{\large{\textbf{Description}}\normalsize\hspace{1ex}\hfill}
In this case, "Operator" is used as a normal Delphi identifier\subsubsection*{\large{\textbf{Properties}}\normalsize\hspace{1ex}\hfill}
\paragraph*{Operator}\hspace*{\fill}

\begin{list}{}{
\settowidth{\tmplength}{\textbf{Description}}
\setlength{\itemindent}{0cm}
\setlength{\listparindent}{0cm}
\setlength{\leftmargin}{\evensidemargin}
\addtolength{\leftmargin}{\tmplength}
\settowidth{\labelsep}{X}
\addtolength{\leftmargin}{\labelsep}
\setlength{\labelwidth}{\tmplength}
}
\begin{flushleft}
\item[\textbf{Declaration}\hfill]
\begin{ttfamily}
public property Operator: string read FOperator write FOperator;\end{ttfamily}


\end{flushleft}
\par
\item[\textbf{Description}]
In this case, "Operator" is used as a normal Delphi identifier, not as an ObjFpc keyword. PasDoc should tolerate this, for compatibility with Delphi and with FPC in {\$}mode delphi.

\end{list}
\section{Functions and Procedures}
\subsection*{:=}
\begin{list}{}{
\settowidth{\tmplength}{\textbf{Description}}
\setlength{\itemindent}{0cm}
\setlength{\listparindent}{0cm}
\setlength{\leftmargin}{\evensidemargin}
\addtolength{\leftmargin}{\tmplength}
\settowidth{\labelsep}{X}
\addtolength{\leftmargin}{\labelsep}
\setlength{\labelwidth}{\tmplength}
}
\begin{flushleft}
\item[\textbf{Declaration}\hfill]
\begin{ttfamily}
Operator := (C : TMyType2) z : TMyType;\end{ttfamily}


\end{flushleft}
\end{list}
\subsection*{+}
\begin{list}{}{
\settowidth{\tmplength}{\textbf{Description}}
\setlength{\itemindent}{0cm}
\setlength{\listparindent}{0cm}
\setlength{\leftmargin}{\evensidemargin}
\addtolength{\leftmargin}{\tmplength}
\settowidth{\labelsep}{X}
\addtolength{\leftmargin}{\labelsep}
\setlength{\labelwidth}{\tmplength}
}
\begin{flushleft}
\item[\textbf{Declaration}\hfill]
\begin{ttfamily}
Operator + (c: TMyType; c1: TMyType) c2: TMyType;\end{ttfamily}


\end{flushleft}
\end{list}
\subsection*{-}
\begin{list}{}{
\settowidth{\tmplength}{\textbf{Description}}
\setlength{\itemindent}{0cm}
\setlength{\listparindent}{0cm}
\setlength{\leftmargin}{\evensidemargin}
\addtolength{\leftmargin}{\tmplength}
\settowidth{\labelsep}{X}
\addtolength{\leftmargin}{\labelsep}
\setlength{\labelwidth}{\tmplength}
}
\begin{flushleft}
\item[\textbf{Declaration}\hfill]
\begin{ttfamily}
Operator - (c: TMyType; c1: TMyType) c2: TMyType;\end{ttfamily}


\end{flushleft}
\end{list}
\subsection*{*}
\begin{list}{}{
\settowidth{\tmplength}{\textbf{Description}}
\setlength{\itemindent}{0cm}
\setlength{\listparindent}{0cm}
\setlength{\leftmargin}{\evensidemargin}
\addtolength{\leftmargin}{\tmplength}
\settowidth{\labelsep}{X}
\addtolength{\leftmargin}{\labelsep}
\setlength{\labelwidth}{\tmplength}
}
\begin{flushleft}
\item[\textbf{Declaration}\hfill]
\begin{ttfamily}
Operator * (c: TMyType; i: integer) c2: TMyType;\end{ttfamily}


\end{flushleft}
\end{list}
\subsection*{/}
\begin{list}{}{
\settowidth{\tmplength}{\textbf{Description}}
\setlength{\itemindent}{0cm}
\setlength{\listparindent}{0cm}
\setlength{\leftmargin}{\evensidemargin}
\addtolength{\leftmargin}{\tmplength}
\settowidth{\labelsep}{X}
\addtolength{\leftmargin}{\labelsep}
\setlength{\labelwidth}{\tmplength}
}
\begin{flushleft}
\item[\textbf{Declaration}\hfill]
\begin{ttfamily}
Operator / (A, B: TMyType): TMyType;\end{ttfamily}


\end{flushleft}
\end{list}
\subsection*{**}
\begin{list}{}{
\settowidth{\tmplength}{\textbf{Description}}
\setlength{\itemindent}{0cm}
\setlength{\listparindent}{0cm}
\setlength{\leftmargin}{\evensidemargin}
\addtolength{\leftmargin}{\tmplength}
\settowidth{\labelsep}{X}
\addtolength{\leftmargin}{\labelsep}
\setlength{\labelwidth}{\tmplength}
}
\begin{flushleft}
\item[\textbf{Declaration}\hfill]
\begin{ttfamily}
Operator ** (A, B: TMyType): TMyType;\end{ttfamily}


\end{flushleft}
\end{list}
\subsection*{=}
\begin{list}{}{
\settowidth{\tmplength}{\textbf{Description}}
\setlength{\itemindent}{0cm}
\setlength{\listparindent}{0cm}
\setlength{\leftmargin}{\evensidemargin}
\addtolength{\leftmargin}{\tmplength}
\settowidth{\labelsep}{X}
\addtolength{\leftmargin}{\labelsep}
\setlength{\labelwidth}{\tmplength}
}
\begin{flushleft}
\item[\textbf{Declaration}\hfill]
\begin{ttfamily}
operator = (const c, d: TMyType) : boolean;\end{ttfamily}


\end{flushleft}
\end{list}
\subsection*{{$<$}}
\begin{list}{}{
\settowidth{\tmplength}{\textbf{Description}}
\setlength{\itemindent}{0cm}
\setlength{\listparindent}{0cm}
\setlength{\leftmargin}{\evensidemargin}
\addtolength{\leftmargin}{\tmplength}
\settowidth{\labelsep}{X}
\addtolength{\leftmargin}{\labelsep}
\setlength{\labelwidth}{\tmplength}
}
\begin{flushleft}
\item[\textbf{Declaration}\hfill]
\begin{ttfamily}
operator {$<$} (const c, d: TMyType) : boolean;\end{ttfamily}


\end{flushleft}
\end{list}
\subsection*{{$>$}}
\begin{list}{}{
\settowidth{\tmplength}{\textbf{Description}}
\setlength{\itemindent}{0cm}
\setlength{\listparindent}{0cm}
\setlength{\leftmargin}{\evensidemargin}
\addtolength{\leftmargin}{\tmplength}
\settowidth{\labelsep}{X}
\addtolength{\leftmargin}{\labelsep}
\setlength{\labelwidth}{\tmplength}
}
\begin{flushleft}
\item[\textbf{Declaration}\hfill]
\begin{ttfamily}
operator {$>$} (const c, d: TMyType) : boolean;\end{ttfamily}


\end{flushleft}
\end{list}
\subsection*{{$<$}=}
\begin{list}{}{
\settowidth{\tmplength}{\textbf{Description}}
\setlength{\itemindent}{0cm}
\setlength{\listparindent}{0cm}
\setlength{\leftmargin}{\evensidemargin}
\addtolength{\leftmargin}{\tmplength}
\settowidth{\labelsep}{X}
\addtolength{\leftmargin}{\labelsep}
\setlength{\labelwidth}{\tmplength}
}
\begin{flushleft}
\item[\textbf{Declaration}\hfill]
\begin{ttfamily}
operator {$<$}= (const c, d: TMyType) : boolean;\end{ttfamily}


\end{flushleft}
\end{list}
\subsection*{{$>$}=}
\begin{list}{}{
\settowidth{\tmplength}{\textbf{Description}}
\setlength{\itemindent}{0cm}
\setlength{\listparindent}{0cm}
\setlength{\leftmargin}{\evensidemargin}
\addtolength{\leftmargin}{\tmplength}
\settowidth{\labelsep}{X}
\addtolength{\leftmargin}{\labelsep}
\setlength{\labelwidth}{\tmplength}
}
\begin{flushleft}
\item[\textbf{Declaration}\hfill]
\begin{ttfamily}
operator {$>$}= (const c, d: TMyType) : boolean;\end{ttfamily}


\end{flushleft}
\end{list}
\subsection*{or}
\begin{list}{}{
\settowidth{\tmplength}{\textbf{Description}}
\setlength{\itemindent}{0cm}
\setlength{\listparindent}{0cm}
\setlength{\leftmargin}{\evensidemargin}
\addtolength{\leftmargin}{\tmplength}
\settowidth{\labelsep}{X}
\addtolength{\leftmargin}{\labelsep}
\setlength{\labelwidth}{\tmplength}
}
\begin{flushleft}
\item[\textbf{Declaration}\hfill]
\begin{ttfamily}
operator or (const c,d:TMyType) : TMyType;\end{ttfamily}


\end{flushleft}
\end{list}
\subsection*{and}
\begin{list}{}{
\settowidth{\tmplength}{\textbf{Description}}
\setlength{\itemindent}{0cm}
\setlength{\listparindent}{0cm}
\setlength{\leftmargin}{\evensidemargin}
\addtolength{\leftmargin}{\tmplength}
\settowidth{\labelsep}{X}
\addtolength{\leftmargin}{\labelsep}
\setlength{\labelwidth}{\tmplength}
}
\begin{flushleft}
\item[\textbf{Declaration}\hfill]
\begin{ttfamily}
operator and (const c,d:TMyType) : TMyType;\end{ttfamily}


\end{flushleft}
\end{list}
\subsection*{xor}
\begin{list}{}{
\settowidth{\tmplength}{\textbf{Description}}
\setlength{\itemindent}{0cm}
\setlength{\listparindent}{0cm}
\setlength{\leftmargin}{\evensidemargin}
\addtolength{\leftmargin}{\tmplength}
\settowidth{\labelsep}{X}
\addtolength{\leftmargin}{\labelsep}
\setlength{\labelwidth}{\tmplength}
}
\begin{flushleft}
\item[\textbf{Declaration}\hfill]
\begin{ttfamily}
operator xor (const c,d:TMyType) : TMyType;\end{ttfamily}


\end{flushleft}
\end{list}
\end{document}
